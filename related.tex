%************************************************************************
\section{Related Work}
\label{sec-RelWork}
%************************************************************************
Model ensemble methods have been applied in disciplines ranging from statistics
to AI (e.g.,~\citep{breiman1996stacked,clemen1989combining,perrone1992soft,wolpert1992stacked}),
and it has been shown that an ensemble is often more accurate than any single
model in the ensemble~\citep{maclin2011popular}.

The idea behind model ensemble is to build a predictive model by integrating multiple models to achieve better performance than could be obtained from the individual ones~\citep{maclin2011popular,rokach2010ensemble}. Roughly speaking, an ensemble
method is a supervised learning-based approach to discover which combination of
model has the best predictive power~\cite{kuncheva2003measures}.

The most popular ensemble algorithms are Bagging and Boosting, which are meta-algorithms
that pool decisions from multiple classifiers.
%
Proposed in 1996 by Leo Breiman, Bagging~\citep{breiman1996stacked}, which stands for
Bootstrap Aggregating, is a \textit{bootstrap}-based ensemble method~\citep{efron1994introduction}.
It applies a majority vote from classifiers trained on bootstrap samples (obtained by
random sampling with replacement) of the training data.

Boosting~\citep{freund1996experiments,schapire1990strength} iteratively learns weak
classified; producing a weighted sum of the results of weak classifiers. The training set
used for each member of the series is chosen based on the performance of the earlier
classifiers in the series. Note that, conversely to Boosting, the resampling of the
training set in Bagging is not dependent on the performance of the earlier classifiers.
Many different kinds of boosting algorithms have been proposed in the past: the
award winning Adaboost: Adaptive boosting~\citep{freund1996experiments}; LPBoost:
Linear Programming Boosting~\citep{demiriz2002linear} is a margin-maximizing classification
algorithm with boosting; BrownBoost~\cite{freund2001adaptive} is a boosting algorithm
that may be robust to noisy datasets; and LogitBoost~\cite{friedman2000additive} approach
which casts the AdaBoost algorithm into a statistical framework.

To the best of our knowledge, model ensemble methods have not been applied to
diagnostics before.

In the context of software fault localization using spectrum-based fault
localization~\citep{abreu2007accuracy}, approaches have been proposed to combine
several heuristics. In~\citep{wang2011search} a search-based algorithm to combine
the existing heuristics. In~\citep{xuan2014learning}, it is proposed a machine
learning-based approach to combine multiple diagnostic rankings metrics. An approach
based in data fusion to heuristic combination was proposed in~\citep{lo2014fusion}.

Another body of related work is research related to choosing the right level of
modeling abstraction. As an example, in the model-based diagnosis literature,
there has been considerable work on diagnostic assumptions and selecting appropriate
models for a diagnostic task~\citep{struss1992s}. The work proposed in~\citep{de2007dynamic}
focuses primarily on assumptions associated with choosing domain abstractions.

There has been considerable research on structural abstraction~\citep{chittaro2004hierarchical,hamscher1990xde}
where groups of components are combined to form larger systems to reduce computational
complexity. The work in~\citep{sachenbacher2005task} describes how the task can be
used to partition the value of a variable into the qualitative values needed to solve a
task. In~\citep{torta2003automatic} presents another
approach to partition the value of a variable into qualitative ranges to reduce
complexity when there is limited observability of the variables.

TODO: we need a sentence on why this is different.
