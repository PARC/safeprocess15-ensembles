\section{Introduction}
%
Model-based diagnosis \citep{dekleer87diagnosing} uses system models
and sensor data to compute diagnostic hypotheses. These hypotheses
have a range of applications such as decision-making
\cite{feldman13genius}, repair, reconfiguration, troubleshooting,
testing, and many others. While providing many benefits, model-based
diagnosis is expensive due to the need to obtain/construct good system
models. To amortize this high modeling cost, researcher develop and
use component libraries.
\par
A component may have several representations in a component
library. For example, a NAND-gate may be modeled as a system of
non-linear equations that govern the analogue electrical laws of the
gate, or its linear approximation or with the simple Boolean
expression $o \leftrightarrow \neg(i_1 \wedge i_2)$. Although one may
say that the best choice of a component model is the one that
represents physics best (in the case of the NAND-gate, this would be
the analogue electrical model), experiments show that the result of
this choice is sometimes hard to predict and dependent on the
diagnostic context (i.e., on the use of the component models in the
final model).
\par
There can be no formal framework for successfully generating a model
of the ``correct'' fidelity. The most common approach is to manually
test different models to examine trade-offs.
\par
In this article we propose the novel approach of using component model
ensembles consisting of multiple models of differing fidelity, for
diagnostics inference. Model ensembles have been used successfully in
machine learning \citep{brown2010ensemble,dietterich2000ensemble}, but
have not been adopted in diagnostics inference.
\par
The main idea of this article is to use a test-set of diagnostic
scenarios for learning the optimal system model. The test-cases in the
test-set can be artificially generated (e.g., by simulation) and
contain a representative set of likely faults. The algorithm we
propose chooses these component models that optimize some (weighted)
diagnostic metrics such as diagnostic accuracy (which is dual of
classification errors), isolation time, or computational complexity
\cite{feldman10empirical}. The output of the algorithm is a system
model composition that can be later used for on-line diagnosis.
\par
To illustrate the usability of our algorithm, consider the diagnostic
model of a crane. The model would contain parts such as electrical
motors and drives and a Programmable Logic Controller
(PLC). Non-linear electromechanical-models are most appropriate for
the choice of the moving parts, however modeling the PLC with
non-linear equations would result in a suboptimal diagnosis due to
high complexity. Further, high simulation accuracy does not
necessarily translate to high diagnostic accuracy. The algorithm we
propose would run a few diagnostic scenarios and would discard the
high-fidelity PLC model to use the computationally simpler
Boolean/qualitative/state-machine components.
\par
We illustrate the working of our algorithm on a dynamic system used
widely in literature. This system consists of three tanks connected
with valves. Even for such a small system the output of the algorithm
is non-intuitive. Last, the experiments in this paper make us believe
that the choice of modeling abstraction depends to a large extent on
the model topology and cannot be preconceived during the design of the
component library.
